\documentclass[]{scrartcl}

\usepackage[T1]{fontenc}

\usepackage[unicode=true,pdfusetitle,
bookmarks=true,bookmarksnumbered=false,bookmarksopen=false,
breaklinks=true,pdfborder={0 0 0},pdfborderstyle={},backref=false,colorlinks=true]
{hyperref}
\hypersetup{linkcolor=blue,citecolor=blue,urlcolor=blue}
\usepackage[backend=biber, style=authoryear, maxbibnames=99, dashed=false]{biblatex}
\setlength\bibitemsep{2\itemsep}
\addbibresource{aleurocanthus.bib}

%opening
\title{Annotated Bibliography for \textit{Aleurocanthus spiniferus} and \textit{A. woglumi} in Micronesia and Hawaii}
\author{Aubrey Moore}

\begin{document}

\maketitle

\section{Micronesia}

\subsection{\fullcite{moore2023}}

A leaf sample of \textit{Triphasia trifolia} with a few \textit{Aleurocanthus} nymphs on the lower surface was collected in Yona, Guam on March 22, 2023.
 
Specimens were sent to Dr. Panagiotis Milonas, Benaki Phytopathological Institute, Greece. He reports that both morphological and molecular tests suggest that the specimens belong to \textit{Aleurocanthus woglumi}, citrus blackfly.

This appears to be a new island record for Guam.

\subsection{\fullcite{moore_list_2016}}

\textit{Aleurocanthus spiniferus}, the orange spiny whitefly is included in the online List of Insects and Mites Attacking Crops in Micronesia as a pest infesting citrus and roses. \textit{A. woglumi} absent from this list.

\subsection{\fullcite{nafus_insect_1997}}
	
Nafus listed \textit{A. spiniferus} as present in the Marianas and as new records for Yap, Chuuk and Kosrai.

\subsection{\fullcite{swezey1942}}


In his extensive insect survey of insects of Guam in 1936, Swezey found only a single whitefly species: the sugarcane whitefly, \textit{Neomasellia bergii}.

\subsection{\fullcite{schreiner_accidental_1986}}

"\textit{\textbf{Aleurocanthus spiniferus}} (Quaintance): Peterson (1955b) reported the introduction of the citrus spiny whitefly in 1951. This insect is widespread in Asia and prior to 1950 was also present in some of the western Caroline Islands. On Guam it became a severe pest of citrus. Several parasite species which had been used to control \textit{Aleurocanthus woglumi} Ashby in Florida, were brought to Guam. \textit{Encarsia smithi} (Silvestri) and \textit{Amitus hesperidum} (Silvestri) became established and have provided successful biological control. \textit{A. spiniferus} recently appeared in the eastern Carolines on Pohnpei and Kosrae."

\subsection{\fullcite{nafus_biological_1989}}

"Since 1950, three whiteflies have been accidentally introduced to Guam. Two of these, \textit{Aleurocanthus spiniferus} (Quaintance) and \textit{Aleurodicus dispersus} Russell, have been the targets of active biological control programs. \textit{A. spiniferus} was successfully controlled by \textit{Amitus hesperidum} Silvestri and \textit{Encarsia smithi} (Silvestri). \textit{E. smithi} was the more important parasite of the two (Peterson, 1955a). Several other species were introduced but failed to establish permanently (Table 1). Biological control of \textit{A. spiniferus} continues to be good, and both \textit{A. hesperidum} and \textit{E. smithi} are still present. \textit{E. smithi} was recently sent from Guam to Kosrae and Pohnpei in the Carolines."

\subsection{\fullcite{nafusEstablishmentEncarsiaSmithi1988}}

\subsection{\fullcite{muniappanEstablishmentEncarsiaSmithi1992}}

\section{Hawaii}

\subsection{\fullcite{beardsley_new_1979}}

\textit{Aleurocanthus spiniferus} was first discovered in Hawaii during 1974.


\subsection{\fullcite{cullineyIntroductionsBiologicalControl}}

\textit{Aleurocanthus woglumi} was first discovered in Hawaii during 1996.

"\textit{\textbf{Aleurocanthus woglumi}} Ashby (Homoptera: Aleyrodidae) (citrus blackfly) First discovered on Oahu in July 1996, this whitefly, the 28th to become established in the state, is known now to be established on most of the main islands (there have been no reports from Niihau or Lanai). Records of \textit{A. woglumi} in Hawaii before 1996 (e.g., Mound and Halsey 1978; CABI/EPPO 1992) are erroneous. The species is widespread and polyphagous, having been recorded from plants in at least 35 families (Mound and Halsey 1978). Citrus spp., however, are the primary hosts, and the only hosts apparently supporting longterm survival of populations (Steinberg and Dowell 1980)."

"Initial surveys on Oahu by HDOA personnel revealed low rates of parasitization of \textit{A. woglumi} by the adventive aphelinid \textit{Encarsia nipponica} Silvestri. Exploration in Central America in the summer of 1998 resulted in the introduction of two other parasitic Hymenoptera, \textit{Amitus hesperidum} Silvestri (Platygasteridae) and \textit{Encarsia perplexa} Huang and Polaszek (at first misidentified as \textit{E. opulenta} [Silvestri]). The former parasitoid is most effective at high host densities whereas the latter species has a higher searching capacity, and is an efficient regulator of whitefly populations at their lower densities (Nguyen and Hamon 1993). These agents were first released on Oahu in the spring of 1999, with releases on neighbor islands following in 2000 and 2001. Just before their release, in March 1999, surveys on Oahu had found an additional species, \textit{E. smithi} (Silvestri) (introduced from Japan in 1974 for control of the orange spiny whitefly \textit{Aleurocanthus spiniferus} [Quaintance]; Nakao and Funasaki 1976), and an already established \textit{E. perplexa} (thought to have been introduced with the blackfly) parasitizing \textit{A. woglumi} at low rates within limited sites (M.W. Johnson, University of Hawaii, personal communication). Although \textit{A. hesperidum} had been recovered from Oahu, its current status is unknown as further recoveries have not been made owing to extremely low host densities. The Guatemalan strain of \textit{E. perplexa} is well established on Oahu and Kauai and providing apparently excellent control of the pest."

\end{document}
